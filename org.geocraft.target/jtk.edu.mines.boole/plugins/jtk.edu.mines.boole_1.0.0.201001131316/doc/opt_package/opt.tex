\documentclass[12pt]{article}
\usepackage{wsh}
\usepackage[dvips]{epsfig}

\begin{document}  
\bibliographystyle{plain}

\markright{An implementation of generalized inversion --- W.S. Harlan}
\title{An implementation of generalized inversion}

\author{William S. Harlan\\
Landmark Graphics; 1805 Shea Center Drive\\
Highlands Ranch, CO 80129, USA}

\date{March 2004}
\def \dv {{\svector d}} 
\def \d0v {{\svector d_0}} 
\def \mv {{\svector m}} 
\def \mzv {{\svector m_0}} 
\def \dmv {{\svector \Delta \mv}} 
\def \ddv {{\svector \Delta \dv}} 
\def \fv {{\svector f}} 
\def \nv {{\svector n}} 
\def \Ct {{\stensor C}} 
\def \Pt {{\stensor P}} 
\def \Cmi {{\Ct}_m^{-1}}
\def \Cni {{\Ct}_n^{-1}}
\def \Ft {{\stensor F}} 
\def \Ftz {{\Ft ( \mzv )}}
\def \FTz {{\Ft^T ( \mzv )}}
\def \Gt {{\stensor G}} 

\maketitle

\section {Introduction}

In my geophysical work, I rely heavily upon a
generalized inversion algorithm called
Gauss-Newton optimization.  The Gauss-Newton
method minimizes objective functions that can
be iteratively approximated by quadratics.
This approach is particularly appropriate for
least-squares inversions of moderately
non-linear transforms.  The package also
contains code for conjugate-gradient and
line-search optimizations.

Over two decades I have implemented this
package four times, in four different
programming languages.  Every feature of this
package has been motivated by practical
application.  This particular Java version is
only a few years old, but I have already used
it for a dozen or more distinct inversion
problems.  I find no preexisting package so
suitable for typical geophysical inversion
problems.

The abstraction is designed to minimize the
burden of supporting new models, data, and
transforms.  Unlike many approaches, we will
not be required to construct and invert large
linear matrices.  We only need to be able to
perform transformations on specific models
and datasets.  The inversion algorithm need
not know any of the implementation details of
those models and transformations.

For each new problem, I must implement two
distinct Java interfaces.  The data and model
are encapsulated by two implementations of
the \texttt{Vect} interface.  The simulation
to be inverted must be encapsulated by
\texttt{Transform} if non-linear, or by
\texttt{LinearTransform} if linear.  
Any constraints or conditioning can be
implemented with one of these interfaces.
Finally,
the model that best explains a particular
dataset is inverted by the
\texttt{Gauss\-Newton\-Solver.solve} method
if non-linear, and 
by \texttt{Quadratic\-Solver.solve} if linear.

\section {Model and data as abstract vectors}

Our optimization algorithm will be defined
only in terms of general vector spaces and
operators on vectors.  Our abstractions of
model and data must first support these
operations.  Notice that this use of the word
vector is not limited to one-dimensional
arrays.  See Luenberger's
\cite{luenbergervec} ``Optimization by Vector
Space Methods.''

Let us write a specific dataset as a vector
$\dv$ and a model as a vector $\mv$.  Each of
these vectors belong to a separate vector
space, with different dimensionality, norms,
and implementations.  Both must support the
same vector operations.

First, a vector can be scaled by a constant
and added to another vector, to produce
another valid vector.  I.e, if $\mv_1$ and
$\mv_2$ are two separate models, and $\alpha$
and $\beta$ are scale factors, then $\alpha
\mv_1 + \beta \mv_2$ is also a model vector.
This vector scaling and addition must be
commutative, associative, and distributive.
Here is the method that must be defined by
the Java interface for an abstract vector.

{\footnotesize \begin{verbatim} 
public interface Vect extends VectConst {
  /** Add a scaled version of another vector to a scaled version of this
      vector.
      @param scaleThis Multiply this vector by this scalar before adding.
      @param scaleOther Multiply other vector by this scalar before adding.
      @param other The other vector to be multiplied.
   */
  public void add(double scaleThis, double scaleOther, VectConst other) ;
  ...
}
\end{verbatim}} 

\noindent We could have defined methods to
scale and add separately, but this method
offered the best compromise between
efficiency and simplicity.  Your
implementation can optimize special cases
where a scale factor is 0 or 1.

\texttt{VectConst} is an interface that
supports only immutable operations.
\texttt{Vect} extends \texttt{VectConst} and
add methods that modify the state of the
vector.

Next we define a dot product (or inner
product) for a normed vector space.  In
vector notation, any two vectors from the
same space can be dotted to produce a scalar:
$\mv_1 \cdot \mv_2$.  This dot product is
usually a simple Cartesian dot product of all
internal parameters.  Any non-Cartesian
distances are separated into a separate
``inverse covariance'' operator.  The
magnitude or norm of a vector is given by $\|
\mv \|^2 \equiv \mv \cdot \Cmi \cdot \mv$ The
significance of this covariance will be
clearer once we define an objective function.

Here are the appropriate Java operations:

{\footnotesize \begin{verbatim} 
public interface VectConst extends Cloneable, java.io.Serializable {
  /** Return the Cartesian dot product of this vector with another
      vector (not including any inverse covariance).
      @param other The vector to be dotted.
      @return The dot product.
  */
  public double dot(VectConst other);

  /** This is the dot product of the vector with itself premultiplied 
      by the inverse covariance.  Equivalently,
        Vect vect = (Vect) this.clone();
        vect.multiplyInverseCovariance();
        return this.dot(vect);
  */
  public double magnitude() ;
  ...
}

public interface Vect extends VectConst {
  ...
  /** Optionally multiply a vector by the inverse covariance matrix.
      Also called preconditioning.
      A method that does nothing is equivalent to an identity.
  */
  public void multiplyInverseCovariance();
  ...
}
\end{verbatim}}

\noindent Notice that all \texttt{VectConst}
extends \texttt{Cloneable} and
\texttt{Serializable}.  Copies will
frequently be necessary.  The optimization
will minimize the number of vectors
instantiated simultaneously, to reduce memory
requirements, even if it requires repeating
some less expensive operations.

The \texttt{VectConst.magnitude()} method is
redundant, but it is necessary to avoid
creating extra copies of the model.  If your
vectors are small, you can use a simple
implementation with a clone, as shown in the
comment.

Serializable is not strictly necessary, but
often highly desirable with creating
collections of vectors.  For example, you can
use the wrapper class \texttt{VectCache} to
create a collection of \texttt{Vect}'s that
are cached to disk.  \texttt{VectCache} is
itself a \texttt{Vect}.  You may also want to
implement a transform that distributes work
to other nodes.  If neither of these
situations are the case, you need not override
the default serialization.

Use the static method
\texttt{VectUtil.test(VectConst vect)} to
test your implementation of all these
methods.

We have now covered all mandatory operations
required for our generalized inversion.  Some
optional operations such as hard constraints
and conditioning are also supported and will
be discussed later.

\section {Abstraction of non-linear
simulation}

Next we need an abstraction of the
calculations that simulate our data from a
particular model.  The inversion algorithm is
not interested in implementation details, but
only in the application of the transform.  If
our data are encapsulated as a vector $\dv$
and our model as $\mv$, then we can write our
simulation as $\dv = \fv ( \mv )$.

With just a forward transform (simulation),
no inversion could efficiently estimate the
model that best explains a particular
dataset.  The inversion would only be able to
try different random models and compare the
results, without any suggestion on how best
to modify the model.  Our abstraction of a
transform requires two more operations.

First, we must know how a small perturbation
of a model affects the data.  If we add a
small perturbation $\dmv$ to a reference
model $\mzv$, then we would like to know the
resulting perturbation to the data $\ddv$.
We must be able to implement a linear
transform $\Ftz$ so that

\begin{equation}
\ddv = \Ftz  \cdot \dmv \approx \fv (\mzv + \dmv) - \fv ( \mzv ) .  
\end{equation}

\noindent The operator $\Ftz$ is linear if
the following equality holds

\begin{eqnarray}
\Ftz \cdot (\alpha \dmv_1 + \beta \dmv_2)
  \equiv \alpha ~ \Ftz \cdot \dmv_1 + \beta ~ \Ftz \cdot \dmv_2 \\ 
~\mbox{for all}~ \alpha, \beta, \dmv_1, ~\mbox{and}~ \dmv_2 .
\nonumber 
\end{eqnarray}

\noindent I always attempt to parameterize my
model so that this linearization is as good
as possible.  The better the linearization,
the better the convergence.

Finally, we will need a transpose $\Ft^T
(\mzv)$ of the linear transform.  A transpose
is defined by the equality

\begin{equation} 
\ddv \cdot [ \Ftz \cdot \dmv ] 
\equiv [\FTz \cdot \ddv ] \cdot \dmv ~ \mbox{for all} ~\ddv~
\mbox{and} ~\dmv.  
\end{equation}

\noindent The transpose is known as
``back-projection'' in tomography and as
``back-propagation'' in neural networks.

In addition to vectors for the data and
model, our generic inversion will require us
to implement an interface for this transform.
In Java, the three essential operations look
like

{\footnotesize \begin{verbatim} 
public interface Transform {
  /** Non-linear transform: data = f(model).
      @param data Output.  Initial values are ignored.
      @param model Input. Unchanged.
   */
  public void forwardNonlinear(Vect data, VectConst model);

  /** A linearized approximation of the forward transform
      for a small perturbation (model) to a reference model (modelReference).
      The output data must be a linear function of the model perturbation.
      Linearized transform:
        data = F model ~= f(model + modelReference) - f(modelReference)
      @param data Output.  Initial values are ignored.
      @param model Perturbation to reference model.
      @param modelReference The reference model for the linearized operator.
   */
  public void forwardLinearized(Vect data, VectConst model,
                                VectConst modelReference);

  /** The transpose of the linearized approximation of the forward transform
      for a small perturbation (model) to a reference model (modelReference):
      model = F' data.  Add the result to the existing model.
      @param data Input for transpose operation.
      @param model Output.  The transpose will be added to this vector.
      @param modelReference The reference model for the linearized operator.
   */
  public void addTranspose(VectConst data, Vect model,
                           VectConst modelReference);
  ...
}
\end{verbatim}} 

\noindent In addition, there is an optional
inverse Hessian method that can be used to
speed convergence.  We will discuss this
method later.

Use the static method
\texttt{VectUtil.getTransposePrecision} to
ensure that your transpose satisfies the
dot-product definition.

We now have all the properties required for a
generalized inversion of a transform to
obtain a model vector that best explains a
data vector.

\section {A damped least-squared objective
function}

The damped least squares objective function
behaves well for geophysical problems that
are both over-deter\-mined and
under-deter\-mined.  When over-deter\-mined,
errors in data will be distributed as
uniformly as possible.  When
under-deter\-mined, components of the model
will be suppressed if they do not improve the
fit with data significantly.  For motivation
of damped least squares as a stochastic
optimization see Menke \cite{menke}, Jaynes
\cite{jaynes}, or Box and Tiao
\cite{boxandtiao}.

We assume that our data $\dv$ are a
non-linear function $\fv(\mv)$ of our model
$\mv$.  The noise vector $\nv$ will be any
component of our data that we cannot explain:

\begin{equation}
\dv = \fv ( \mv ) + \nv.  
\end{equation}

\noindent Assume the noise $\nv$ and model
$\mv$ are sampled from separate Gaussian
distributions.  The model that best explains
a specific dataset is the one that minimizes
this objective function.

\begin{equation}
\label{eq:obj}
\min_{\mv} ~ \{ [\dv - \fv (\mv) ] \cdot \Cni \cdot [\dv - \fv (\mv) ]
            + \mv \cdot \Cmi \cdot \mv \} .
\end{equation}

\noindent The covariance matrices describe
expected correlations between each Gaussian
parameter in a vector.  (Assume without loss
of generality that the mean of the model and
noise are zero.  We can adjust our transform
accordingly.)

This objective function will be easier to
optimize if covariances are as diagonal as
possible.  I discuss such manipulation of the
quadratic objective function in the paper
``Regularization by Model Parameterization''
\cite{harlan-model}.  I strongly recommend
you examine this paper as well.

You need only damp the magnitude of the model
slightly to assure a stable inverse.  If your
noise and model samples are completely
stationary and independent, your
\texttt{Vect.multiply\-Inverse\-Covariance()}
methods need only multiply by a scalar.  I
first anticipate reasonable physical
magnitudes for the noise and for the model.
If both noise and model attain their most
reasonable values, then the two squared
penalty terms should be equal in the
objective function (\ref{eq:obj}).  Instead I
adjust covariances so that model parameters
can attain magnitudes 10 or 100 times their
most reasonable values before the two terms
are equal.  This minimal damping is
sufficient to guarantee a stable inverse
without distorting the solution
unnecessarily.  (Note that both covariances
can be multiplied by the same arbitrary scale
factor without affecting the final solution.)

A Gauss-Newton algorithm iteratively
approximates the objective function as a
quadratic function of a perturbation of the
model:

\begin{eqnarray}
\label{eq:qobj}
\min_{\dmv} ~  [\dv - \fv (\mzv) - \Ftz \cdot \dmv ] \cdot
 \Cni \cdot [\dv - \fv (\mzv) - \Ftz \cdot \dmv ] \nonumber \\
           + (\mzv + \dmv)  \cdot \Cmi \cdot (\mzv + \dmv).
\end{eqnarray}

\noindent To minimize a completely quadratic
objective function, we can use the
well-understood conjugate-gradient algorithm
(see Gill et al \cite{gill} or Luenberger
\cite{luenberger}).  The best perturbation of
the model $\dmv$ is is actually a linear
function of the remaining error in the data
$\dv - \fv (\mzv)$.

We might be tempted simply to add the
perturbation $\dmv$ to the reference model
$\mzv$, but non-linearity in the transform
can cause divergence.  Instead the algorithm
solves for a single scale factor $\alpha$ to
multiply the perturbation $\dmv$ before
addition.

\begin{eqnarray}
\min_{\alpha} ~  [\dv - \fv (\mzv + \alpha \dmv) ] \cdot
 \Cni \cdot [\dv - \fv (\mzv + \alpha \dmv) ] \nonumber \\
           + (\mzv + \alpha \dmv)  \cdot \Cmi \cdot (\mzv + \alpha \dmv).
\end{eqnarray}

\noindent Often in geophysical problems, the
original objective function (\ref{eq:obj})
has less curvature than the quadratic
approximation.  In such a case, the estimated
scale factor $\alpha$ can be expected to be
less than one.

To search for a single scalar in a known
range, the line-search algorithm
(\texttt{ScalarSolver}) minimizes the number
of evaluations of the original objective
function.  I use a line-search that searches
for a parabolic minimum with hyper-linear
convergence.  If convergence is poor because
of strong non-linearity, the search resorts
to a golden section with at least linear
convergence.

\section {Constraints and conditioning}

The \texttt{Vect} and \texttt{Transform}
interfaces include some optional methods to
apply constraints and to improve convergence.

Often our nonlinearity is due entirely to
hard constraints upon our model.  For
example, seismic velocities may be
constrained to a strict range of positive
values.

We also often have a partial inverse of our
forward transform, perhaps from an analytic
derivation that assumes our data are more
evenly sampled.  Or we may know that a
forward and transpose operation amplifies
certain components of the data.  We should be
able to use such information to improve
convergence.

If you have a simple hard constraint, such as
an inequality, define it in the method
\texttt{Vect.constrain()}.  An empty
implementation does nothing.  This method
does not affect the conjugate-gradient
optimization of the quadratic approximation.
Instead this method is applied after adding a
perturbation to the reference model.  The
method is called repeatedly during the line
search for a best scale factor for the
perturbation.

You may instead wish to limit the degrees of
freedom available to an update, without
similarly constraining your reference model.
Your model and your updates need not share
the same implementations of the \texttt{Vect}
interface.  You need only convert from one to
the other by implementing the
\texttt{Vect.project} method, which has the
same arguments as the \texttt{Vect.add}
method.

Most users of prepackaged inversions are
familiar with ``preconditioning'' as a method
of improving convergence.  Instead of
attempting to invert a transform like $\dv =
\fv (\mv)$, one inverts a related problem
$\Pt \cdot \dv = \Pt \cdot \fv (\mv)$.  The
preconditioning transform $\Pt$ amplifies
components of the data that should be
inverted first.  Unfortunately, solving this
problem is not the same as solving our
original problem.  Our final distribution of
errors will be different.  We have
effectively defined a new dataset and a new
transform, which of course we can do at any
time without help from the inversion code.

As I mentioned before, my favorite method of
stabilizing and improving convergence is a
reparameterization of the model
\cite{harlan-model}.  This approach
simplifies and diagonalizes the model
covariance operator.

Even after reparameterization, we may want to
encourage certain parts of the model to be
updated first.  We may know that our forward
transform has greater or lesser sensitivity
to certain components of our model.  Both the
\texttt{Vect} and \texttt{Transform}
interfaces allow us to add
``post-conditioning'' that takes advantage of
such knowledge.

The \texttt{Vect} interface has an optional
method \texttt{Vect\urlbr{}.postCondition()}.
If you define an empty implementation for
this method, you will receive the default
behavior, which should be fine.  For better
convergence, this method can apply a linear
filter to the data that enhances components
that should be optimized first, and
suppresses components of lesser importance.
This is the last operation applied to a model
gradient before use in the conjugate-gradient
algorithm.  Your objective function is not
modified, but your perturbations of that
model will improve.  You should be able to
obtain a better model in fewer iterations.

Alternatively, you may have already have an
approximate linear inverse for the linearized
forward transform.  We call this inverse the
``inverse Hessian'' because the Hessian is
the tensor curvature of the quadratic
approximation.  

You define an approximate inverse Hessian by implementing the 
\texttt{Transform\urlbr{}.multiply\-Inverse\-Hessian()}
method.  Leaving a default empty implementation
is equivalent to assuming an identity.

For a given linearization
$\Ftz$, the inverse Hessian $\Gt ( \mzv )$
should be the partial inverse

\begin{eqnarray}
\Gt ( \mzv ) \approx [ \FTz \cdot \Ftz ]^{-1}, ~ \mbox{or} \\
\Gt ( \mzv ) \cdot [ \FTz \cdot \Ftz] \approx {{\stensor I}}.
\end{eqnarray}

\noindent As long as the product $\Gt \cdot
\Ft^T \cdot \Ft$ is more diagonal, or better
balanced along the diagonal, this inverse
Hessian will improve convergence.  In
tomography such a filter is called ``hit
count balancing'' and in slant-stacks a ``rho
filter.'' Again, this operator implies no
change in the objective function, just in the
way we perturb our solution.

One last optional operation is useful for
very noisy data:
\texttt{Transform\urlbr{}.adjust\-Robust\-Errors(Vect
data\-Error)}.  This operation does imply a
change to our objective function, though a
minor one.  This method will be passed an
estimate of the errors in fitting the data
--- the original data minus the simulation of
the data with the best current model.  Your
method may detect that a small subset of your
data has exceptionally large errors.  Rather
than allow a few bad data to distort your
result, your method can clip or scale back
these errors.  Do not change the overall
variance of the errors more than necessary.
(If you are familiar with L1 optimization,
you can also use this method to divide all
errors by their previous magnitudes, to
iteratively approximate a median fit to your
data.)

\section {The Gauss Newton solver}

Most everything unique about your inversion
problem is described by your implementation
of the \texttt{Vect} and \texttt{Transform}
interfaces.  For an inversion with all
features described previously, you should
call this static method of
\texttt{Gauss\-Newton\-Solver}:

{\footnotesize \begin{verbatim} 
public class GaussNewtonSolver {
  public static Vect solve (VectConst data,
                            VectConst referenceModel,
                            VectConst perturbModel,   // optional
                            Transform transform,     
                            boolean dampOnlyPerturbation,
                            int conjugateGradIterations,
                            int lineSearchIterations,
                            int linearizationIterations,
                            double lineSearchError,
                            Monitor monitor)         // optional
...
}
\end{verbatim}} 

The return value of this method is the model
that best explains an instance of your data,
passed as the first argument \texttt{data}.

The second argument \texttt{referenceModel}
is the model that should be used as the
initial guess of your solution.  Often you
can initialize this model to a zero
magnitude.  The returned solution will be
always be a revised instance of this
reference model.

If you want to perturb the reference model
with an instance of a different class, then
provide a \texttt{perturbModel}.  If
\texttt{perturbModel} is null, then instances
of the reference model will be used for
perturbations.  The \texttt{project} method
of the reference model should accept the
perturbation as an argument.  Perturbations
should have fewer degrees of freedom than the
reference model, because projection will lose
any additional degrees of freedom.  The
initial state of this vector is ignored.

Your \texttt{Transform} should be able to
model your data from the reference model with
\texttt{forwardNonlinear}, and should be able
to use an instance of the reference or
perturbed model as perturbations in the
\texttt{forwardLinearized} and
\texttt{addTranspose} methods.

If \texttt{dampOnlyPerturbations} is true,
then the objective function will only
minimize differences between your reference
model and your new model.  Otherwise, the
absolute \texttt{Vect.magnitude()} of your
model will be minimized.

Remaining parameters let you control how much
computational effort you are willing to
expend on the solution.

Each quadratic approximation should be
minimized with at least a few iterations of
conjugate-gradients.  The parameter
\texttt{conjugateGradIterations} should have
a minimum of 3 iterations and a typical value
of 4 or 5.  If your transform is relatively
inexpensive, then indulge in a few more.
Each iteration results in an additional
evaluation of the linearized forward and
transpose transform.  If this parameter has a
value of 1, then your optimization will be
equivalent to non-linear steepest-descent,
which is notorious for poor convergence when
the Hessian has a poorly balanced diagonal.

To scale the conjugate-gradient perturbation,
we must perform a line search.  The parameter
\texttt{lineSearchIterations} controls the
maximum number of evaluations of the
non-linear forward transform to be used for
this search.  Because of the hyper-linear
convergence, you can expect fairly good
optimization of this scale factor with as few
as 12 iterations.  Typically I prefer a safer
value of 20, beyond which you are unlikely to
see an improvement.  If you specify 0, then
you will get a default scale factor of 1.
Such a choice might converge well on a
transform that was almost entirely linear to
begin with.  A value of 1 might be a good
choice if the your only non-linearity results
from hard constraints on your model from the
\texttt{Vect.constrain()} method.

The number of iterations actually used for
the line search depends on the required
precision of the scale factor.  The parameter
\texttt{lineSearchError} is the acceptable
fractional error in the estimated scale
factor.  I typically use a value of 0.001 or
smaller.

The parameter
\texttt{linearizationIterations} controls the
outermost loop of reapproximating the
objective function as a quadratic.  Each of
these iterations multiplies the number of
conjugate-gradient and line search
evaluations required.  The number of
quadratic approximations should be fairly
small, unless your transform is strongly
non-linear.  I typically choose a minimum
value of 3 unless I can afford more.

If you wish to save the results of each
linearized iteration, then you can construct
your own loop: set
\texttt{linearization\-Iterations} to 1 and
use the solution from each iteration as the
reference model for the next.  Managing your
own outer loop will not increase the
computational cost significantly.

To get a good feel for the difficulty of your
problem, see if increasing the number of
iterations or precision improves your
solution significantly.

To track the progress of your inversion, you
can optionally pass a non-null implementation
of the \texttt{Monitor} interface.  This
interface defines a single method
\texttt{public void report(double fraction)}
which will be called regularly with the
current fraction of work done.  Values range
from 0 at the beginning to 1 when all work is
done.  To print the progress to a java
\texttt{Logger}, use the existing
implementation in \texttt{LogMonitor}.

\section {Simpler conjugate-gradient and
scalar solvers}

If your transform is entirely linear, you
need only use a conjugate-gradient solver.
You can implement the simpler
\texttt{LinearTransform} interface that
contains one instead of two forward
transforms.  Use this solver instead:

{\footnotesize \begin{verbatim} 
public class QuadraticSolver {
...
  public static Vect solve (VectConst data,
                            VectConst referenceModel,
                            LinearTransform linearTransform,
                            boolean dampOnlyPerturbation,
                            int conjugateGradIterations,
                            Monitor monitor) 
...
}
\end{verbatim}} 

All other parameters have the same meaning as
before.  No linearization or line-search is
necessary, so the corresponding parameters
have disappeared.

The conjugate-gradient algorithm itself is
contained in the
\texttt{Quad\-ra\-tic\-Sol\-ver} class.  To
use this class directly, you must construct
the normal equations of your least-squares
problem, including a Hessian operator.  For
inversion problems, this form is much less
convenient than the
\texttt{Quadratic\-Solver.solve} method
above.  But if your objective function is
already expressed as a simple quadratic, then
you might want to use this lower-level class.

You might also have occasion to estimate a
single parameter that minimizes some
arbitrary function.  You can use the
\texttt{ScalarSolver} class directly by
implementing the single method in the
\texttt{ScalarSolver.Function} interface.
This is the same algorithm used internally by
the Gauss-Newton algorithm for the
line-search of a scale factor.

\section {Conclusion}

I have described all methods available in the
interfaces available for solving a
generalized least-squares inversion problem.
Methods exist for testing the internal
consistency of your implementations.  I think
you will find a large number of inversion
problems can be solved by this framework.
When the simulation is too non-linear for
this framework, it is also likely to be
difficult for any framework to optimize well.
You would be well advised to attempt a
reparameterization of your model.

\bibliography{wsh}

\end{document}

% $Id: inv.tex,v 1.1 2004/04/05 15:49:12
harlan Exp $

